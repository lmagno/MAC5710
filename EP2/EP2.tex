\documentclass[a4paper]{article}

\usepackage[T1]{fontenc}
% \usepackage[utf8]{inputenc}
\usepackage[brazilian]{babel}				% Brazilian portuguese language/hyphenation
\usepackage{amsmath,amsfonts,amsthm,amssymb}% Math packages
\usepackage{graphicx}						% Enable pdflatex
\usepackage{indentfirst}
\usepackage{fullpage}
\usepackage{listings}
\usepackage[boxed]{algorithm2e}
% \usepackage{subcaption}						% Images side-by-side
% \usepackage{bm}
% \usepackage{minted}
% \usepackage{fontspec}
% \usepackage[hidelinks]{hyperref}
% \usepackage{natbib}
% \usepackage{siunitx}
% \usepackage{pgffor}
% \usepackage{tkz-graph}                      % Graph drawing
% \usepackage{setspace}                       % Double spacing
% \usepackage{newfloat}
% \usepackage{ifthen}

\newcommand{\ttt}{\texttt}
\SetKwInput{Entrada}{Entrada}
\SetKwInput{Saida}{Saída}
\SetKwInput{Dados}{Dados}
\SetKwInput{Resultado}{Resultado}
\SetKw{Ate}{até} \SetKw{KwRetorna}{retorna}
\SetKw{Retorna}{retorna}
\SetKw{Continua}{continua}
\SetKwBlock{Inicio}{início}{fim}
\SetKwIF{Se}{SenaoSe}{Senao}{se}{então}{senão se}{senão}{fim se}
% \SetKwSwitch{Selec}{Caso}{Outro}{selecione}{faça}{caso}{senão}{fim caso}fim selec
\SetKwFor{Para}{para}{faça}{fim para}
\SetKwFor{ParaPar}{para}{faça em paralelo}{fim para}
\SetKwFor{ParaCada}{para cada}{faça}{fim para cada}
\SetKwFor{ParaTodo}{para todo}{faça}{fim para todo}
\SetKwFor{Enqto}{enquanto}{faça}{fim enqto}
\SetKwRepeat{Repita}{repita}{até}

\title{{\scshape\small MAC5710 - Estruturas de Dados e sua Manipulação} \\ Exercício-Programa 2}
\author{Lucas Magno \\ 7994983}
\date{}

\begin{document}
\maketitle

\section{Introdução}
    Este trabalho consiste em, a partir de um \emph{dicionário}, um arquivo com sequência de palavras, determinar um maior conjunto de anagramas dentre tais palavras. Um dicionário tem a forma:

    \begin{lstlisting}[captionpos=b,caption={Exemplo de dicionário},escapeinside={\%*}{*)}]
        A
        a
        aa
        aal
        aalii
        aam
        Aani
        aardvark
        aardwolf
        Aaron
        %*\vdots *)
        zymotoxic
        zymurgy
        Zyrenian
        Zyrian
        Zyryan
        zythem
        Zythia
        zythum
        Zyzomys
        Zyzzogeton
    \end{lstlisting}
    Para tal foi desenvolvido um programa em C, cujos detalhes serão discutidos a seguir. Também foi implementado um gerador de dicionários, a fim de testar o programa, discutido mais adiante.

\newpage
\section{Algoritmo}
    O algoritmo para encontrar um maior conjunto de anagramas consiste em, para cada string de um dado dicionário, calcular uma \emph{hash} (tal que todos os anagramas de uma palavra têm a mesma \emph{hash}) e colocar a string na fila do nó correspondente a tal \emph{hash} em uma árvore binária de busca. Ao longo do processo, basta manter um registro de uma maior fila encontrada até então e ao final teremos uma maior fila, um maior conjunto de anagramas, do dicionário.

    Devido à nececessidade de se verificar um grande número de strings contra determinadas chaves (\emph{hashes}), algoritmo faz uso de uma árvore binária de busca $B$, que permite tal busca em tempo logarítmico, cujas chaves e valores de cada nó são
    \begin{description}
        \leftskip=2em
        \item[Chave:] um vetor \texttt{letters} de 26 inteiros, onde cada elemento representa a contagem de uma determinada letra em uma palavra.\\
            \texttt{letters[0]} é o número de vezes que a letra $a$ aparece,\\
            \vdots\\
            \texttt{letters[25]} é o número de vezes que a letra $z$ aparece.

        \item[Valor:] uma fila de strings, tais que todas têm a mesma contagem de letras que a chave.
    \end{description}

    Onde a comparação entre duas chaves $l_1$ e $l_2$ é feita da seguinte forma:
    \begin{align*}
        \begin{cases}
            l_1 = l_2,\quad &\text{se}\ l_1[i] = l_2[i],\text{ para todo } i = 0,\dots, 25\\
            l_1 < l_2,\quad &\text{se}\ l_1[j] = l_2[j],\ j = 0, \dots, i-1\\
                            &\text{ e } l_1[i] < l_2[i] \text{ para algum } i = 0,\dots,25\\
            l_1 > l_2,\quad &\text{se}\ l_1[j] = l_2[j],\ j = 0, \dots, i-1\\
                            &\text{ e } l_1[i] > l_2[i] \text{ para algum } i = 0,\dots,25\\
        \end{cases}
    \end{align*}

    Assim, ao irmos adicionando as strings lidas do dicionário $D$ às filas de cada chave correspondente, garantimos que cada uma dessas filas é um conjunto de anagramas, pois contém strings com exatamente as mesmas letras. Portanto, após ler todo o dicionário, uma fila com maior tamanho é um maior conjunto de anagramas em $D$. A seguir o pseudocódigo do programa implementado.

    \begin{algorithm}[h]
        \SetKwData{B}{$B$}\SetKwData{D}{$D$}
        \SetKwFunction{Length}{length}
        \Entrada{\D dicionário}
        \Saida{Fila com as strings que pertencem a um maior conjunto de anagramas em \D}
        \BlankLine

        \B $\leftarrow$ árvore binária de busca vazia\;
        $qmax \leftarrow NULL$\;
        \Enqto{não chegou ao final de \D}{
            ler uma string $s$ de \D\;
            $ls \leftarrow$ a forma \emph{lowered} de $s$\tcp*{letras minúsculas}
            $letters \leftarrow$ um vetor com a contagem das letras em $ls$\;
            % \tcp{letters[0] é a quantidade de vezes que a letra $a$ aparece em $ls$}
            \BlankLine
            \eSe{$letters$ está em $B$}{
                adiciona $s$ à fila correspondente a esta chave\;
            }{
                cria um nó correspondente a esta chave em $B$ e coloca $s$ na nova fila\;
            }
            $q \leftarrow$ a fila correspondente a $letters$ em $B$\;
            \Se{\Length{q} > \Length{qmax}}{
                $qmax \leftarrow q$\;
            }

        }

        \Retorna{qmax}\;
        \caption{Busca por um maior conjunto de anagramas em um dicionário.}
    \end{algorithm}

\newpage
\section{Gerador}
    Para testar o programa, também foi implementado um gerador de dicionários, baseado na função \texttt{npermutations}, que gera as $n$ primeiras (ou todas, se não existirem mais que $n$) permutações únicas em ordem lexicográfica de um conjunto de letras, determinado por uma sequência \texttt{letters} de letras.\footnote{A implementação é um pouco diferente, utilizando um vetor \texttt{letters} de contagem como definido anteriormente e mais argumentos, por questões de eficiência, mas a lógica é a mesma.}

    A função funciona recursivamente, gerando permutações únicas de subsequências de \texttt{letters} e as concatenando com cada letra única restante, como descrito no algoritmo a seguir.

    \begin{algorithm}[h]
        \SetKwFunction{Length}{length}
        \SetKwFunction{Enfileira}{enfileira}
        \SetKwFunction{Desenfileira}{desenfileira}
        \SetKwFunction{npermutations}{npermutations}
        \Entrada{
            \begin{description}
                \leftskip=2em % Indentation
                \item[letters] uma sequência de letras;
                \item[n] um inteiro.
            \end{description}
        }
        \Saida{As $n$ primeiras permutações únicas de $letters$ em ordem lexicográfica.}
        \BlankLine

        $P \leftarrow$ uma fila de strings vazia\;
        \BlankLine
        \Se{\Length{letters} = 1}{
            $s \leftarrow$ a única letra em $letters$\;
            \Enfileira{$P$, $s$}\;
            \Retorna $P$\;
        }
        \BlankLine
        \ParaCada{letra única c em letters em ordem alfabética}{
            $Q \leftarrow \npermutations{letters - c,n}$\tcp*{remove uma instância de $c$ em $letters$}
            \Enqto{\Length{Q} > 0}{
                $s \leftarrow \Desenfileira{Q}$\;
                \Enfileira{P, c$\cdot$s}\tcp*{concatenação}
                \BlankLine
                \Se{\Length{P} = n}{
                    \Retorna{P}\;
                }

            }
        }
        \BlankLine
        \Retorna{P}\;
        \caption{Função \texttt{npermutations}.}
    \end{algorithm}

    A partir desta função podemos construir de fato o gerador de dicionários, que consiste em:
    \begin{enumerate}
        \item Gerar uma contagem de letras \ttt{letters} aleatória com um dado tamanho máximo \ttt{wordlen};
        \item Calcular um dado número \ttt{setlen} de permutações dessas letras e salvar como o maior conjunto de anagramas \ttt{qmax};
        \item Inserir \ttt{qmax} em uma árvore binária de busca \ttt{B} com a chave \ttt{letters};
        \item Enquanto ainda precisarmos gerar mais palavras sortear outra contagem de letras \ttt{letters} e, se esta não está em \ttt{B}, calcular um número menor que \ttt{setlen} de permutações e as salvar em \ttt{B} com a chave \ttt{letters}.
    \end{enumerate}

    A seguir é dado o algoritmo mais detalhadamente.

    \newpage
    \begin{algorithm}[h]
        \SetKwFunction{Min}{min}
        \SetKwFunction{Rand}{rand}
        \SetKwFunction{Imprime}{imprime}
        \SetKwFunction{Insere}{insere}
        \SetKwFunction{LettersRandom}{lettersrandom}
        \SetKwFunction{Length}{length}
        \SetKwFunction{Enfileira}{enfileira}
        \SetKwFunction{Desenfileira}{desenfileira}
        \SetKwFunction{npermutations}{npermutations}
        \Entrada{
            \begin{description}
                \leftskip=2em % Indentation
                \item[wordlen] o tamanho máximo das palavras;
                \item[setlen] o tamanho máximo dos conjuntos de anagramas;
                \item[dictlen] o número de palavras no dicionário;
                \item[file] o nome arquivo de saída.
            \end{description}
        }
        \Saida{Um dicionário com $dictlen$ palavras de tamanho máximo $wordlen$, cujo único maior conjunto de anagramas $qmax$ tem tamanho no máximo $setlen$, escrito no arquivo de saída $file$. Além disso, imprime $qmax$ para a \texttt{STDOUT}.}
        \BlankLine

        $B \leftarrow $ uma árvore binária de busca vazia\;
        $letters \leftarrow \LettersRandom{wordlen}$\;
        $qmax \leftarrow$ \npermutations{letters, setlen}\;
        $B[letters] \leftarrow qmax$\tcp*{insere $qmax$ em $B$ com chave $letters$}
        $setlen \leftarrow \Length{qmax}$\tcp*{o conjunto gerado pode ser menor que $setlen$}
        \BlankLine

        \Imprime{qmax}\;
        \Enqto{\Length{qmax} > 0}{
            $s \leftarrow \Desenfileira{qmax}$\;
            \Imprime{file, s}\;
        }
        \BlankLine

        $restantes \leftarrow dictlen - setlen$\tcp*{número de palavras que ainda temos que gerar}
        \Enqto{restantes > 0}{
            $len \leftarrow$ \Rand{$[1, wordlen]$}\tcp*{número aleatório no intervalo $[1, wordlen]$}
            $letters \leftarrow \LettersRandom{len}$\;

            \Se{letters está em B}{\Continua\;}
            \BlankLine

            $m \leftarrow $ \Rand{$[1,\Min{restantes, setlen-1}]$}\;
            $q \leftarrow$ \npermutations{letters, m}\;
            $B[letters] \leftarrow q$\;
            \BlankLine

            \Enqto{\Length{q} > 0}{
                $s \leftarrow$ \Desenfileira{q}\;
                \Imprime{file, s}\;
            }
            $restantes \leftarrow restantes - \Length{q}$\;
        }
        \caption{Gerador de dicionários.}
    \end{algorithm}

    Onde \ttt{lettersrandom(n)} é uma função que retorna uma contagem aleatória de letras com tamanho $n$.

\newpage
\section{Compilação e Uso}
    Para compilar os programas, basta executar os seguintes comandos:
    \begin{description}\leftskip=3em
        \item[Programa:] \ttt{make}
        \item[Gerador:] \ttt{make gen}
    \end{description}
    que criam os executáveis \ttt{EP2} e \ttt{gen}, respectivamente.

    O programa \ttt{EP2} deve ser invocado da forma
    \begin{lstlisting}
        ./EP2 dicionario.txt
    \end{lstlisting}
    e o resultado é mostrado na tela.

    Já o gerador é invocado na forma
    \begin{lstlisting}
        ./gen wordlen setlen dictlen output.txt
    \end{lstlisting}
    onde \ttt{wordlen}, \ttt{setlen} e \ttt{dictlen} são inteiros positivos como definido anteriormente e \ttt{output.txt} é o arquivo de saída onde será gravado o dicionário. O programa então imprime para a tela o maior conjunto de anagramas gerado.

\section{Resultados}
    Para o dicionário \ttt{dicionario.txt} fornecido como exemplo, obtemos a seguinte saída
    \begin{lstlisting}
        Biggest set of anagrams:
        Queue (11 elements):
            angor
            argon
            goran
            grano
            groan
            nagor
            Orang
            orang
            organ
            rogan
            Ronga
    \end{lstlisting}
    e cuja execução levou $0.27\,s$.

    A seguir são dados outros exemplos gerados, que foram executados num computador com a seguinte configuração (saída do programa \ttt{neofetch}):
    \begin{lstlisting}
              eeeeeeeeeeeeeeeee
           eeeeeeeeeeeeeeeeeeeeeee         ----------------
         eeeee  eeeeeeeeeeee   eeeee       OS: elementary OS 0.4 Loki x86_64
       eeee   eeeee       eee     eeee     Model: Z97-D3H
      eeee   eeee          eee     eeee    Kernel: 4.4.0-77-generic
     eee    eee            eee       eee   Uptime: 14 hours, 21 mins
     eee   eee            eee        eee   Packages: 2300
     ee    eee           eeee       eeee   Shell: fish 2.2.0
     ee    eee         eeeee      eeeeee   Resolution: 1920x1080
     ee    eee       eeeee      eeeee ee   DE: Pantheon
     eee   eeee   eeeeee      eeeee  eee   WM: Mutter(Gala)
     eee    eeeeeeeeee     eeeeee    eee   Theme: Arc [GTK2/3]
      eeeeeeeeeeeeeeeeeeeeeeee    eeeee    Icons: Paper [GTK2/3]
       eeeeeeee eeeeeeeeeeee      eeee     Terminal: mc
         eeeee                 eeeee       CPU: Intel i5-4690K (4) @ 3.900GHz
           eeeeeee         eeeeeee         GPU: NVIDIA GeForce GTX 750
              eeeeeeeeeeeeeeeee            Memory: 4915MiB / 15925MiB
    \end{lstlisting}

    \newpage
    \subsection{\tt ./gen 4 10 100000 output.txt}
        Tempos de execução:
        \begin{align*}
            \ttt{gen:}\quad &0.10\, s\\
            \ttt{EP2:}\quad &0.04\, s
        \end{align*}

        Saída de \ttt{gen}:
        \begin{lstlisting}
            Biggest set of anagrams:
            Queue (10 elements):
                dhsy
                dhys
                dshy
                dsyh
                dyhs
                dysh
                hdsy
                hdys
                hsdy
                hsyd
        \end{lstlisting}

        Saída de \ttt{EP2}:
        \begin{lstlisting}
            Biggest set of anagrams:
            Queue (10 elements):
                dhsy
                dhys
                dshy
                dsyh
                dyhs
                dysh
                hdsy
                hdys
                hsdy
                hsyd
        \end{lstlisting}

    \newpage
    \subsection{\tt ./gen 8 10 1000000 output.txt}
        Tempos de execução:
        \begin{align*}
            \ttt{gen:}\quad &0.97\, s\\
            \ttt{EP2:}\quad &0.63\, s
        \end{align*}

        Saída de \ttt{gen}:
        \begin{lstlisting}
            Biggest set of anagrams:
            Queue (10 elements):
                aijlpuvz
                aijlpuzv
                aijlpvuz
                aijlpvzu
                aijlpzuv
                aijlpzvu
                aijlupvz
                aijlupzv
                aijluvpz
                aijluvzp
        \end{lstlisting}

        Saída de \ttt{EP2}:
        \begin{lstlisting}
            Biggest set of anagrams:
            Queue (10 elements):
                aijlpuvz
                aijlpuzv
                aijlpvuz
                aijlpvzu
                aijlpzuv
                aijlpzvu
                aijlupvz
                aijlupzv
                aijluvpz
                aijluvzp
        \end{lstlisting}

    \newpage
    \subsection{\tt ./gen 20 20 1000000 output.txt}
        Tempos de execução:
        \begin{align*}
            \ttt{gen:}\quad &0.98\, s\\
            \ttt{EP2:}\quad &0.44\, s
        \end{align*}

        Saída de \ttt{gen}:
        \begin{lstlisting}
            Biggest set of anagrams:
            Queue (20 elements):
                defggijjklmmorttwxyz
                defggijjklmmorttwxzy
                defggijjklmmorttwyxz
                defggijjklmmorttwyzx
                defggijjklmmorttwzxy
                defggijjklmmorttwzyx
                defggijjklmmorttxwyz
                defggijjklmmorttxwzy
                defggijjklmmorttxywz
                defggijjklmmorttxyzw
                defggijjklmmorttxzwy
                defggijjklmmorttxzyw
                defggijjklmmorttywxz
                defggijjklmmorttywzx
                defggijjklmmorttyxwz
                defggijjklmmorttyxzw
                defggijjklmmorttyzwx
                defggijjklmmorttyzxw
                defggijjklmmorttzwxy
                defggijjklmmorttzwyx

        \end{lstlisting}

        Saída de \ttt{EP2}:
        \begin{lstlisting}
            Biggest set of anagrams:
            Queue (20 elements):
                defggijjklmmorttwxyz
                defggijjklmmorttwxzy
                defggijjklmmorttwyxz
                defggijjklmmorttwyzx
                defggijjklmmorttwzxy
                defggijjklmmorttwzyx
                defggijjklmmorttxwyz
                defggijjklmmorttxwzy
                defggijjklmmorttxywz
                defggijjklmmorttxyzw
                defggijjklmmorttxzwy
                defggijjklmmorttxzyw
                defggijjklmmorttywxz
                defggijjklmmorttywzx
                defggijjklmmorttyxwz
                defggijjklmmorttyxzw
                defggijjklmmorttyzwx
                defggijjklmmorttyzxw
                defggijjklmmorttzwxy
                defggijjklmmorttzwyx

        \end{lstlisting}

\section{Conclusão}
    Como visto pelos resultados, a saída de ambos programas coincidem, o que sugere que a implementação do \ttt{EP2} esteja correta e que a saída para o exemplo \ttt{dicionario.txt} também. Além disso, os tempos de execução estiveram dentro do esperado para o tamanho do problema, o que por sua vez sugere boa eficiência da implementação, embora seriam necessários mais testes (e em outros computadores) para se confirmar isso.

    Uma complicação que poderia ocorrer é devido à árvore binária de busca utilizada, cuja implementação é a mais simples, então não há controle de altura e portanto poderíamos ter situações em que seu desempenho é linear, e não logarítmico como esperado. Isso não ocorreu nos testes pois os dicionários considerados têm as palavras em ordem aleatória (em relação à comparação entre chaves da árvore binária). Nos casos em que os dicionários estão ordenados pelas chaves, o ideal seria utilizar alguma árvore binária balanceada. 
\end{document}
